\documentclass[]{jlreq}

\usepackage{graphicx}
\usepackage[pdfencoding=auto]{hyperref}
\usepackage{amsmath,amssymb}
\usepackage{bm}
\usepackage{booktabs}
%\usepackage{subfig}
\usepackage{pifont}
\usepackage{url}
\usepackage{cite}
\usepackage{ulem}
\usepackage{siunitx}
\usepackage{float}
\usepackage{tcolorbox}
\tcbuselibrary{breakable}
\usepackage{cancel}
\usepackage{color}
\renewcommand{\CancelColor}{\color{red}}
\renewcommand{\abstractname}{}

\usepackage{tikz}
\usetikzlibrary{shadows}
\usetikzlibrary{calc}
%\usepackage{circuitikz}

\usepackage{luatexja-fontspec}
\renewcommand{\figurename}{Fig.~}
\renewcommand{\tablename}{Table~}

\hypersetup{
  colorlinks=false, % リンクに色をつけない設定
  %bookmarks=true, % 以下ブックマークに関する設定
  bookmarksnumbered=true,
  pdfborder={0 0 0},
  bookmarkstype=toc
}

\begin{document}
\title{Time-domain and Frequency-domain Signals}
\author{吉本伸一}
\maketitle
\tableofcontents
\clearpage

\section{Fourier級数展開}
周期が$T_0$の周期関数$f(t)$をFourier級数展開すると
%
\begin{equation}
  f(t) = \sum_{n=-\infty}^{\infty} c_n e^{j n \omega_0 t}
\end{equation}
%
\begin{equation}
  c_n = \frac{1}{T_0} \int_{-T_0/2}^{T_0/2} f(t) e^{- j n \omega_0 t} dt
\end{equation}
%
ただし、$\omega_0 = 2\pi/T_0$とおいた。
%
\section{Fourier変換}
\begin{equation}
  \mathcal{F}[f(t)] = F(\omega) = \int_{-\infty}^{\infty} f(t) e^{-j\omega t} dt
\end{equation}
\begin{equation}
  \mathcal{F}^{-1}[F(\omega)] = f(t) = \frac{1}{2\pi}\int_{-\infty}^{\infty} F(\omega) e^{j\omega t} d\omega
\end{equation}

\section{デルタ関数}
\begin{equation}
  \int_{-\infty}^{\infty} \delta (x) dx = 1
\end{equation}
%
\begin{equation}
  \int_{-\infty}^{\infty} f(x) \delta (x) dx = f(0)
\end{equation}
%
\begin{equation}
  \int_{-\infty}^{\infty} f(x) \delta (x - a) dx = f(a)
\end{equation}
%
\section{デルタ関数のFourier変換}
%
\begin{equation}
  \mathcal{F}[\delta (t)] = \int_{-\infty}^{\infty} \delta (t) e^{- j \omega t} dt = 1 
\end{equation}
%
\begin{equation}
  \mathcal{F}[\delta (t - t_0)] = \int_{-\infty}^{\infty} \delta (t - t_0) e^{- j \omega t} dt = e^{-j \omega t_0}
  \label{fd2}
\end{equation}
%
\section{くし型関数のFourier変換}
周期が$T_0$のくし型関数は以下のよう表せる。
%
\begin{equation}
  \delta_T (t) = \sum_{n=-\infty}^{\infty} \delta (t - n T_0 )
\end{equation}
%
$\delta_T (t)$は周期$T_0$の周期関数なので、フーリエ級数展開でき、$\omega_0 = 2\pi / T_0$とすると
%
\begin{align}
  \delta_T(t) &= \sum_{n = -\infty}^{\infty} \frac{1}{T_0} \left[\int_{-T_0/2}^{T_0/2}\delta (t) e^{-j n \omega_0 t} dt\right] 
  e^{i n \omega_0 t} \\
              &= \frac{1}{T_0} \sum_{n = -\infty}^{\infty} e^{j n \omega_0 t} 
\end{align}
%
となる。なお、区間$[-T_0/2, T_0/2]$においては$\delta_T(t)$は$\delta(t)$とみなせることを使った。ここで、式(\ref{fd2})より
%
\begin{align}
  e^{j n \omega_0 t} &=  \int_{-\infty}^{\infty} \delta (\omega - n \omega_0) e^{j \omega t} d\omega \\ 
      &= \frac{1}{2\pi} \int_{-\infty}^{\infty} 2\pi \delta (\omega - n \omega_0) e^{j \omega t} d\omega \\ 
      &= \mathcal{F}^{-1}[2\pi \delta(\omega - n\omega_0)] 
\end{align}
%
したがって、くし型関数のFourier変換は
%
\begin{align}
  \mathcal{F}[\delta_T(t)] &= \mathcal{F}\left[\frac{1}{T_0} \sum_{n = -\infty}^{\infty} e^{j n \omega_0 t}\right] \\ 
      & = \frac{1}{T_0} \sum_{n = -\infty}^{\infty} \mathcal{F}[\mathcal{F}^{-1}[2\pi \delta(\omega - n\omega_0)]] \\ 
      & = \frac{2 \pi}{T_0} \sum_{n = -\infty}^{\infty} \delta(\omega - n\omega_0)\\ 
      &= \omega_0 \sum_{n = -\infty}^{\infty} \delta(\omega - n\omega_0)
\end{align}
%
\section{Multi-bunch Spectrum}
リングにM個のバンチが均等に配置されている場合、ビームの信号は
%
\begin{equation}
  f(t) = \sum_{k=-\infty}^{\infty} \sum_{n=0}^{M-1}\delta (t - k T_0 -n T_0 / M)
\end{equation}
%
と表すことができる。くし型関数と同様に考えると、$f(t)$は周期$T_0/M$の周期関数で、区間$[-T_0/2M, T_0/2M]$においては$f(t)$は$\delta(t)$とみなせるから
%
%
\begin{align}
  f(t) &= \sum_{n = -\infty}^{\infty} \frac{M}{T_0} \left[\int_{-T_0/2M}^{T_0/2M}\delta (t) e^{-j n M \omega_0 t} dt\right] 
  e^{j n M \omega_0 t} \\
              &= \frac{M}{T_0} \sum_{n = -\infty}^{\infty} e^{j n M \omega_0 t} 
\end{align}
%
したがって、
%
\begin{align}
  \mathcal{F}[f(t)] &= \mathcal{F}\left[\frac{M}{T_0} \sum_{n = -\infty}^{\infty} e^{j n M \omega_0 t}\right] \\ 
      & = \frac{M}{T_0} \sum_{n = -\infty}^{\infty} \mathcal{F}[\mathcal{F}^{-1}[2\pi \delta(\omega - n M \omega_0)]] \\ 
      & = \frac{2 \pi M}{T_0} \sum_{n = -\infty}^{\infty} \delta(\omega - n M \omega_0)\\ 
      &= M \omega_0 \sum_{n = -\infty}^{\infty} \delta(\omega - n M \omega_0)
\end{align}
%
\begin{thebibliography}{9}
  \bibitem{Hiramatsu}
  平松成範, 加速器のビームモニター (Beam Instrumentaion for Accelerrators), KEK Internal 2004-4 A.
\end{thebibliography}
%
\end{document}