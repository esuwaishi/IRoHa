\documentclass[]{jlreq}

\usepackage{graphicx}
\usepackage[pdfencoding=auto]{hyperref}
\usepackage{amsmath,amssymb}
\usepackage{bm}
\usepackage{booktabs}
%\usepackage{subfig}
\usepackage{pifont}
\usepackage{url}
\usepackage{cite}
\usepackage{ulem}
\usepackage{siunitx}
\usepackage{float}
\usepackage{tcolorbox}
\tcbuselibrary{breakable}
\usepackage{cancel}
\usepackage{color}
\renewcommand{\CancelColor}{\color{red}}
\renewcommand{\abstractname}{}

\usepackage{tikz}
\usetikzlibrary{shadows}
\usetikzlibrary{calc}
%\usepackage{circuitikz}

\usepackage{luatexja-fontspec}
\renewcommand{\figurename}{Fig.~}
\renewcommand{\tablename}{Table~}

\hypersetup{
  colorlinks=false, % リンクに色をつけない設定
  %bookmarks=true, % 以下ブックマークに関する設定
  bookmarksnumbered=true,
  pdfborder={0 0 0},
  bookmarkstype=toc
}

\begin{document}
\title{Synchrotron Motion}
\author{吉本伸一}
\maketitle
\tableofcontents
\clearpage

\begin{abstract}
  静電加速器の電界強度は、電界破壊や加速管の長さによって制限されるため、静電加速器は主に低エネルギーの加速器に使用されてきた。一方,共振状態で動作する低損失の高周波(rf)空洞を用いれば,$V \sin(\phi_s + \omega t)$の加速電圧を供給することができる.本章では,rf加速電圧波が存在する場合の粒子ダイナミクスを研究します。
  
  シンクロトロンとベータトロンの両方の振動に対して6次元ハミルトニアンを導出することができるが(Chap.2, Sec.IX参照)、ここでは簡単のために、回転周波数とエネルギーゲインの関係のみに基づいてシンクロトロンのハミルトニアンを導出することにする。この形式論は、シンクロトロン運動とベータトロン運動の間の本質的なつながりを欠いているが、シンクロトロンの位相空間座標の選択を単純化することができる。
  
  回転周期$T_0$、運動量$p_0$でrf位相$\phi = \phi_s$に同期した粒子を同期粒子と呼びます。同期粒子はrf空洞を通過するごとに$e V \sin \phi_s$のエネルギーを得たり失ったりします。通常、同期粒子がすべての磁石の中心を通過する閉軌道上を移動するように、理想的な磁場が配置されている。異なるベータトロン振幅を持つ粒子は、この理想的な閉軌道の周りをベータトロン運動する。
  
  ビームバンチはわずかに異なる運動量を持つ粒子で構成されている。ここで、Dは分散関数、$\delta = (p - p_0)/p_0$は運動量の分数偏差を表す。エネルギー獲得量は高周波磁場と粒子の到着時間の同期に敏感に依存するので、同期した粒子を加速すると、運動量がわずかに異なる粒子はどうなるのだろうか?
  
  シンクロトロン運動の位相集束原理は、McMillanとVekslerによって発見された\cite{McMillan}。回転周波数$f$が、より高い運動量の粒子に対してより高い場合、すなわち$df / d\delta > 0$の場合、より高いエネルギーの粒子は、より早くrfギャップに到着する、すなわち$\phi < \phi_s$となる。したがって、rf波の同期位相が、$0 < \phi_s< \pi/2$であるように選択された場合、より高いエネルギー粒子は、rfギャップからより少ないエネルギーゲインを受ける。同様に、低エネルギー粒子は同じrfギャップに遅れて到着し、同期粒子よりも多くのエネルギーを得ることになります。このプロセスにより、シンクロトロン運動の位相安定性が得られる。$df/d\delta < 0$の場合、位相安定性には$\pi/2 < \phi_s < \pi$が必要である。
  
  位相安定性の発見は、「シンクロトロン」と呼ばれるすべての現代の高エネルギー加速器への道を開き、半世紀にわたる研究開発を経て、現在でも現代の加速器の礎となっている。位相安定性のない粒子加速は、コッククロフト・ウォルトン、ファン・デ・グラーフ、ベータトロンなどの低エネルギー加速器に限られる。さらに、束ねたビームを短くしたり、長くしたり、組み合わせたり、重ねたりして、RF操作スキームを使って多くの高度なアプリケーションを実現することができます。位相空間の体操は、高エネルギー貯蔵リングの操作に不可欠なツールとなっている。
  
  本章では、シンクロトロン運動のダイナミクスを研究する。セクションIでは、さまざまな位相空間座標におけるシンクロトロンの運動方程式を導出する。セクションIIでは、不変トーラスが一定のハミルトニアン値に対応する断熱的なシンクロトロン運動を扱う。III節では、RFの位相と振幅の変調、双極子場の誤差によるシンクロ・ベータトロン結合、地面の振動などから生じるシンクロトロンの運動の摂動を研究する。Sec.IVでは、ハミルトニアンが不変ではない遷移エネルギー付近の非断熱的なシンクロトロン運動を扱う。セクションVでは、ビームの入射、取り出し、スタッキング、バンチローテーション、位相変位の加速、ダブルRFシステムやバリアRFシステムでのビーム操作などを研究している。セクションVIでは、RF空洞設計の基本的な側面を扱う。セクションVIIでは、縦方向の集団的不安定性について紹介する。Sec.VIIIでは、リニアックの紹介をしています。
\end{abstract}


\section{Longitudinal Equation of Motion}
シンクロトロンは、荷電粒子が偏向電磁石によって曲げられ、高周波加速空洞によって加速される円形加速器である (Fig.~\ref{synchrotron}) 。線形加速器とは異なり、粒子は閉軌道を周回し、加速空洞で繰り返し加速される。
%
\begin{figure}[hbt]
  \begin{center}
    \includegraphics[width=12cm,clip]{figs/synchrotron.pdf}
    \caption{シンクロトロンの概略図.}
   \label{synchrotron}
  \end{center}
\end{figure}

粒子の運動方程式は、
%
\begin{equation}
  \frac{d\bm{p}}{dt} = e (\bm{E} + \bm{v}\times \bm{B})
\end{equation}
%
となるが、動径方向の運動を考えると
%
\begin{equation}
  \frac{m v^2}{\rho} = |\bm{F_r}| = e v B
\end{equation}
%
したがって、粒子の運動量と磁場の間には次の関係が成り立つ。
%
\begin{equation}
  p = mv = e \rho B
  \label{momentum_B}
\end{equation}
%
粒子の運動量は加速空洞で加速される毎に増加するので、曲率半径と周回軌道を一定に保つためには、(\ref{momentum_B}) より、偏向電磁石の磁場$B$は運動量に比例して増加しなければならない。また、回転周波数$f_{rev}$も運動量と共に増加するので、粒子と加速電圧の同期を維持するために加速周波数$f_{RF}$も追随する必要がある。

\section{The Synchrotron Mapping Equation}
Hamilton形式では、高周波電場は加速器内で一様に分布していると考えた。しかし、実際には、高周波空洞はシンクロトロンの短い部分に局在しているので、シンクロトロンの運動は次のシンプレクティック写像式によってより現実的に記述される。
%
\begin{equation}
  \begin{split}
    \delta_{n+1} &= \delta_{n} + \frac{eV}{\beta^2 E}(\sin\phi_n - \sin\phi_s) \\
    \phi_{n+1} &= \phi_{n} + 2\pi h \eta(\delta_{n+1})\delta_{n+1}
    \label{map_eq}
  \end{split}
\end{equation}
%
マッピング方程式の物理学は次のように視覚化できます。まず、粒子はrf空洞をn回目に通過する際にエネルギーを得たり失ったりします。そして、rf位相$\phi_{n+1}$は新しいオフモーメンタム座標$\delta_{n+1}$に依存します。$(\phi_n, \delta_n)$から$(\phi_{n+1}, \delta_{n+1})$へのマッピングのヤコビアンは1に等しいので、マッピングは位相空間の面積を保存することになる。

式(\ref{map_eq})では、RF空洞を加速器内の1つの塊状の要素として扱っていることに注意してください。実際には、RF空洞は不均一に分布している可能性があります。また、異なる空洞間のRF位相変化も均一ではないかもしれない。シンクロトロンの動きは通常遅いので、ハミルトニアン形式とマッピング方程式は等価です。マッピング方程式は簡単なので、通常、粒子追跡計算に使用されます。

\section{Evolution of Synchrotron Phase-Space Ellipses}
式(3.18)から得られる軌跡$(\phi, \delta)$で囲まれた位相空間の面積は、エネルギーに依存しません。したがって、式(3.18)は粒子の追跡シミュレーションには使えません。(3.18)は加速中の粒子ビームの追跡シミュレーションには使用できません。加速中、$(\phi, \Delta E / \omega_0)$における位相空間領域は不変である。位相空間座標$(\phi, \Delta E / \omega_0)$に対する位相空間マッピング方程式を使用する必要があります。位相空間領域の断熱減衰は、位相空間座標$(\phi, \Delta E / \omega_0)$を$(\phi, \delta)$に変換することで得られます。

図3.2に示したrfバケットのセパラトリックスは閉曲線である。加速勾配が高い急速循環型シンクロトロンや電子リニアックでは、セパラトリックスは閉曲線ではない。シンクロトロンの位相空間座標$(\phi, \Delta E)の$マッピング方程式は以下の通りである。

\section{トランジション・エネルギー}
加速空洞で加速された粒子は運動量が増加するため、先ほど見たように軌道長の増加と共に、速度も増加する。今、同期粒子の速度を$v_s$とし、同期粒子のローレンツ因子を
%
\begin{equation}
  \gamma_s = \frac{1}{\sqrt{1 - \beta_s^2}}, \quad \beta_s = \frac{v_s}{c}
\end{equation}
%
とすると、(\ref{dv_dp}) より、運動量の変化と速度の変化の間に以下の関係が成り立つ。
%
\begin{equation}
  \frac{\Delta v}{v_s}=\frac{1}{\gamma_s^2}\frac{\Delta p}{p_s}
  \label{delta_v}
\end{equation}
%
このように運動量の変化$\Delta p$によって、速度$\Delta v$と軌道長$\Delta C$が変化するが、それによって回転周期$T$は同期粒子の周期$T_{rev}=C_0/v_s$とすると、
%
\begin{equation}
  T = T_{rev} + \Delta T = \frac{C_0 + \Delta C}{v_s + \Delta v} \notag
\end{equation}
%
と変化する。分母を払って整理すると、(二次の微小量は無視する)
%
\begin{align}
  & (T_{rev} + \Delta T)(v_s + \Delta v) = C_0 + \Delta C \notag \\
  \Longleftrightarrow\quad & \underset{C_0}{\uwave{\cancel{v_s T_{rev}}}} + \Delta v T_{rev} + v_s \Delta T +
  \underset{0}{\uwave{\cancel{\Delta v \Delta T}}}
  = \cancel{C_0} + \Delta C \notag \\
  \Longleftrightarrow\quad & v_s \Delta T = \Delta C- \Delta v T_{rev} \notag \\
  \Longleftrightarrow\quad & \frac{\Delta T}{T_{rev}} = \frac{\Delta C_0}{\underset{C_0}{\uwave{v_s T_{rev}}}} - \frac{\Delta v}{v_s} \notag
\end{align}
%
したがって、$\Delta T$, $\Delta C$, $\Delta v$の関係は以下のようになる。
%
\begin{equation}
  \frac{\Delta T}{T_{rev}} = \frac{\Delta C}{C_0} - \frac{\Delta v}{v_s}
  \label{delta_T}
\end{equation}
%
\begin{equation}
  \frac{\Delta T}{T_{rev}} = \left(\alpha_p - \frac{1}{\gamma_s^2}\right)\frac{\Delta p}{p_s} = \eta_p \delta_p
  \label{deltat_eta_deltap}
\end{equation}
%
ただし、
%
\begin{equation}
  \eta_p \equiv \alpha_p - \frac{1}{\gamma_s^2} = \frac{1}{\gamma_t^2} - \frac{1}{\gamma_s^2}
  \label{alppha_slip}
\end{equation}
%
はphase slip factorと言い、運動量の変化による相対的な回転周期$T$の変化率を表している。
%
運動量圧縮率が正の場合 ($\alpha_p>0$) を考える。このとき、粒子の運動量を増加させると経路長が増加する。 しかし、粒子の運動量がまだ低く、光度よりも十分遅い場合、粒子の運動量の増加による速度の増加は、運動量の増加による経路長の増加を越えることがあり得る。その結果、粒子の回転周期は短くなる ($\eta_p < 0$)。
一方、粒子の運動量が十分高く超相対論的粒子の場合、粒子の速度は既に光速に十分近く、運動量の増加による速度の増加は極僅かである。この場合、経路長の増加が速度の増加を上回り、粒子の回転周期は長くなる ($\eta_p>0$) 。$\eta_p = 0$の時、つまり
%
\begin{equation}
  \gamma_t = \frac{1}{\sqrt{\alpha_p}}
\end{equation}
%
では回転周期は運動量に依存しない。この$\gamma_t$をトランジション・エネルギー (transition energy) と言う。

\vspace{\baselineskip}

\begin{tcolorbox}[title=\textgt{SuperKEKB LERのtransition energy}]
  SuperKEKBの陽電子リングLERの運動量圧縮率は$\alpha_p = 3.2 \times 10^{-4}$なので、
  %
  \begin{equation}
    \gamma_t = \frac{1}{\sqrt{\alpha_p}}= \frac{1}{\sqrt{3.2 \times 10^{-4}}} = 55.9 \notag
  \end{equation}
  %
  陽電子の静止質量は$E_0 = \SI{0.5109989461}{\mega\electronvolt}$だから
  \begin{equation}
    E = \gamma_t E_0 = \SI{28.6}{\mega\electronvolt} \notag
  \end{equation}
  %
  となり、入射器からの入射エネルギー$\SI{4}{\giga\electronvolt}$の方がtransition energyより十分高い。
\end{tcolorbox}

\section{位相安定性の原理}

\begin{figure}[hbt]
  \begin{center}
    \includegraphics[width=15cm,clip]{figs/phase_stability.pdf}
    \caption{空洞電圧と同期粒子 $(p_s)$, より高い運動量を持つ粒子 $(p>p_s)$, より低い運動量 $(p<p_s)$ を持つ粒子の位相関係.}
    \label{phase_stability}
  \end{center}
\end{figure}

ビームの中には多くの粒子があり、色々な運動量を持つ粒子で構成されている。同期粒子と異なる運動量を持つ粒子が加速空洞を通過する時、どのように加速されるかについて考えてみる (Fig.~\ref{phase_stability}) 。

$\eta_p < 0$ の時、同期粒子より大きい運動量を持つ粒子 $(p>p_s$) は、同期粒子より回転周期が短いので、同期粒子より早く加速空洞に到着する。したがって、同期位相$\phi_s$を$0<\phi_s<\pi/2$に選べば、より大きい運動量を持つ粒子は、同期粒子より少ないエネルギーを加速空洞から得ることになる。同様に、同期粒子より小さな運動量を持つ粒子 $(p<p_s)$ は、同期粒子より遅く加速空洞に到達するので、同期粒子より多いエネルギーを加速空洞から得る。この過程を繰り返すことで、同期粒子の持つ運動量から外れた運動量を持つ粒子は同期粒子の周りを安定に運動することが分かる。$\eta_p > 0$ の時は、同期位相を$\pi/2<\phi_s<\pi$に選べば同様にこの運動は安定である。

\section{Difference Equations for Longitudinal Motion in a Synchrotron}
ここでは、粒子の周回毎の縦方向の運動について、これまでと同様にリングに加速空洞は一台だけ設置されている場合を考える。空洞電圧のゼロクロス点からの相対時間を$\tau$、相対位相を$\phi\pmod{2\pi}$とする (Fig.~\ref{coordinates})。今、$n+1$周回目の粒子の空洞への到着時間$\tau(n+1)$と$n$周回目の到着時間$\tau(n)$との差は、粒子の$n+1$周回目の回転周期$T(n+1)$と同期粒子の回転周期$T_{rev}(n+1)$との差に等しいので、
%
\begin{equation}
  \tau(n+1) - \tau(n) = T(n+1) -T_{rev}(n+1)
  \label{tau_T}
\end{equation}
%
\begin{figure}[hhbt]
  \begin{center}
    \includegraphics[width=15cm,clip]{figs/coordinates.pdf}
    \caption{空洞電圧と相対時間$\tau$, 相対位相$\phi$の関係 ($h=1$の場合).}
    \label{coordinates}
  \end{center}
\end{figure}
%
$\phi(n+1) = \omega_{RF}(n+1) \tau(n+1)$から、この関係を用いると、
%
\begin{align}
  \phi(n+1) &=\omega_{RF}(n+1)\{\tau(n) + T(n+1) -T_{rev}(n+1)\} \notag \\
    &= \frac{\omega_{RF}(n+1)}{\omega_{RF}(n)} \phi(n) + \omega_{RF}(n+1)\{T(n+1) -T_{rev}(n+1)\}  \notag
\end{align}
%
(\ref{deltat_eta_deltap})より
%
\begin{equation}
    \frac{T(n+1)-T_{rev}(n+1)}{T_{rev}(n+1)} = \eta \delta_p(n+1)
\end{equation}
%
を用いると、
%
\begin{align}
  \phi(n+1) &= \frac{\omega_{RF}(n+1)}{\omega_{RF}(n)}\phi(n) + \omega_{RF}(n+1)T_{rev}(n+1)\eta \delta_p(n+1) \notag \\
  &=\frac{\omega_{RF}(n+1)}{\omega_{RF}(n)}\phi(n) + 2\pi h \eta \delta_p(n+1)
\end{align}
%
ここで、
%
\begin{equation}
  \frac{\omega_{RF}(n+1)}{\omega_{RF}(n)} \approx 1
\end{equation}
%
という近似を行うと、位相に関して次の関係が成り立つ。
%
\begin{equation}
  \phi(n+1) = \phi(n) + 2\pi h \eta \delta_p(n+1)
\end{equation}

次に、$n+1$周回目の粒子のエネルギー$E(n+1)$は、$E(n)$に位相$\phi(n)$で空洞を通過する時に得られるエネルギーを足し合わせたものだから、
%
\begin{equation}
  E(n+1) = E(n) + e V \sin\phi (n)
\end{equation}
%
同期粒子に関しても同様に考えると、
%
\begin{equation}
  E_s(n+1) = E_s(n) + e V \sin\phi_s
\end{equation}
%
両辺の差を取り、$\Delta E = E - E_s$とすると、
%
\begin{equation}
  \Delta E(n+1) = \Delta E(n) + e V (\sin\phi(n) - \sin\phi_s)
\end{equation}
%
ここで、(\ref{dp_de})より、
%
\begin{equation}
  \delta_p = \frac{\Delta p}{p_s} = \frac{1}{\beta_s^2}\frac{\Delta E}{E_s}
  \label{delta_p}
\end{equation}
%
より、
%
\begin{equation}
  \Delta E(n+1) - \Delta E(n) = \beta^2 E_s \{\delta_p(n+1) - \delta_p(n)\}
\end{equation}
%
したがって、
%
\begin{equation}
  \delta_p(n+1) - \delta_p(n) = \frac{e V}{\beta_s^2 E_s}(\sin\phi(n) -\sin\phi_s)
\end{equation}
%

以上より、シンクロトロンの縦方向の運動について、周回毎の差分方程式は以下のようになる。
%
\begin{align}
  \begin{split}
     &\delta_p(n+1) = \delta_p(n) + \frac{e V}{\beta_s^2 E_s}(\sin\phi(n) -\sin\phi_s) \\
     &\phi(n+1) = \phi(n) + 2\pi h \eta \delta_p(n+1)
    \label{map}
  \end{split}
\end{align}
%
これらの式より、$(\phi(n),\,\delta_p(n))$の値を順次求めることで縦方向の運動を調べることができる (particle tracking simulation)。

\section{Differential Equations for Longitudinal Motion in a Synchrotron}
$\phi$と$\delta_p$のリング一周あたりの変化が十分小さい時、離散的な差分方程式である (\ref{map}) は、連続的な微分方程式に書き換えることができる。この場合、$\phi$と$\delta_p$のリング一周あたりの変化はそれぞれ、近似的に
%
\begin{equation}
  \frac{\delta_p(n+1)-\delta_p(n)}{T_{rev}(n+1)} \approx \frac{d\delta_p}{dt}=\dot{\delta_p},\quad 
  \frac{\phi(n+1)-\phi(n)}{T_{rev}(n+1)} \approx \frac{d\phi}{dt}= \dot{\phi}
\end{equation}
%
と書けるので、(\ref{map}) は
%
\begin{align}
    \begin{split}
      \dot{\delta_p} &= \frac{e V \omega_{rev}}{2\pi \beta_s^2 E_s}(\sin\phi - \sin\phi_s) \\
      \dot{\phi} &= h \omega_{rev} \eta \delta_p
    \end{split}
\end{align}
%
この二つの式をまとめると、
%
\begin{equation}
  \ddot{\phi} = \frac{e V h \eta \omega_{rev}^2}{2\pi\beta_s^2 E_s}(\sin\phi-\sin\phi_s)
  \label{ddot_phi}
\end{equation}
%
$\Delta\phi = \phi - \phi_s$が非常に小さい時、
%
\begin{equation}
  \sin\phi = \sin(\phi_s+\Delta\phi) \approx \sin\phi_s + \cos\phi_s \Delta\phi
\end{equation}
%
より (\ref{ddot_phi}) は
%
\begin{equation}
  \ddot{\Delta\phi} = \frac{e V h \eta \omega_{rev}^2}{2\pi \beta_s^2 E_s} \cos\phi_s \Delta\phi = - \omega_s^2 \Delta\phi
\end{equation}
%
となり、この運動は$\phi_s$の周りの単振動であることが分かる。ここで、この振動の振動数は
%
\begin{equation}
  \omega_s = \sqrt{-\frac{e V h \eta \omega_{rev}^2 \cos\phi_s}{2\pi \beta_s^2 E_s}}
\end{equation}
%
となり、これをシンクロトロン振動数と呼ぶ。
%
\vspace{\baselineskip}

\begin{tcolorbox}[title=\textgt{SuperKEKBのシンクロトロン周波数}, breakable = true]
  SuperKEKB両リングのシンクロトロン振動に関係するパラメータは、

  \vspace{\baselineskip}

  \begin{center}
    \begin{tabular}{l|c|cc|c}
      & & LER   &   HER & units \\ \hline
      Beam Energy & $E_s$ & 4.000 & 7.007 & GeV  \\
      Momentum Compaction & $\alpha_p$ & $3.20\times 10^{-4}$& $4.55\times 10^{-4}$ & \\
      Total Cavity Voltage & $V_c$ & 9.4   &   15.0  &   MV \\
      Energy Loss/turn & $U_s$   & 1.76  &   2.43  & MeV \\
      RF frequency & $f_{RF}$ & \multicolumn{2}{c|}{508.887} & MHz \\
      Harmonic number & $h$       &   \multicolumn{2}{c|}{5120}  &
    \end{tabular}
  \end{center}
  
  \vspace{\baselineskip}

  同期位相$\phi_s$は$U_s = e V_c \sin \phi_s$から求まり
  %
  \begin{equation}
    \sin \phi_s = \frac{U_s}{e V_c}
  \end{equation}

  SuperKEKBの陽電子リングLERの運動量圧縮率は$\alpha_p = 3.2 \times 10^{-4}$なので、
  %
  \begin{equation}
    \gamma_t = \frac{1}{\sqrt{\alpha_p}}= \frac{1}{\sqrt{3.2 \times 10^{-4}}} = 55.9 \notag
  \end{equation}
  %
  陽電子の静止質量は$E_0 = \SI{0.5109989461}{\mega\electronvolt}$だから
  \begin{equation}
    E = \gamma_t E_0 = \SI{28.6}{\mega\electronvolt} \notag
  \end{equation}
  %
  となり、入射器からの入射エネルギー$\SI{4}{\giga\electronvolt}$の方がtransition energyより十分高い。

  SuperKEKBの陽電子リングLERの運動量圧縮率は$\alpha_p = 3.2 \times 10^{-4}$なので、
  %
  \begin{equation}
    \gamma_t = \frac{1}{\sqrt{\alpha_p}}= \frac{1}{\sqrt{3.2 \times 10^{-4}}} = 55.9 \notag
  \end{equation}
  %
  陽電子の静止質量は$E_0 = \SI{0.5109989461}{\mega\electronvolt}$だから
  \begin{equation}
    E = \gamma_t E_0 = \SI{28.6}{\mega\electronvolt} \notag
  \end{equation}
  %
  となり、入射器からの入射エネルギー$\SI{4}{\giga\electronvolt}$の方がtransition energyより十分高い。

  SuperKEKBの陽電子リングLERの運動量圧縮率は$\alpha_p = 3.2 \times 10^{-4}$なので、
  %
  \begin{equation}
    \gamma_t = \frac{1}{\sqrt{\alpha_p}}= \frac{1}{\sqrt{3.2 \times 10^{-4}}} = 55.9 \notag
  \end{equation}
  %
  陽電子の静止質量は$E_0 = \SI{0.5109989461}{\mega\electronvolt}$だから
  \begin{equation}
    E = \gamma_t E_0 = \SI{28.6}{\mega\electronvolt} \notag
  \end{equation}
  %
  となり、入射器からの入射エネルギー$\SI{4}{\giga\electronvolt}$の方がtransition energyより十分高い。

  SuperKEKBの陽電子リングLERの運動量圧縮率は$\alpha_p = 3.2 \times 10^{-4}$なので、
  %
  \begin{equation}
    \gamma_t = \frac{1}{\sqrt{\alpha_p}}= \frac{1}{\sqrt{3.2 \times 10^{-4}}} = 55.9 \notag
  \end{equation}
  %
  陽電子の静止質量は$E_0 = \SI{0.5109989461}{\mega\electronvolt}$だから
  \begin{equation}
    E = \gamma_t E_0 = \SI{28.6}{\mega\electronvolt} \notag
  \end{equation}
  %
  となり、入射器からの入射エネルギー$\SI{4}{\giga\electronvolt}$の方がtransition energyより十分高い。

  SuperKEKBの陽電子リングLERの運動量圧縮率は$\alpha_p = 3.2 \times 10^{-4}$なので、
  %
  \begin{equation}
    \gamma_t = \frac{1}{\sqrt{\alpha_p}}= \frac{1}{\sqrt{3.2 \times 10^{-4}}} = 55.9 \notag
  \end{equation}
  %
  陽電子の静止質量は$E_0 = \SI{0.5109989461}{\mega\electronvolt}$だから
  \begin{equation}
    E = \gamma_t E_0 = \SI{28.6}{\mega\electronvolt} \notag
  \end{equation}
  %
  となり、入射器からの入射エネルギー$\SI{4}{\giga\electronvolt}$の方がtransition energyより十分高い。

  SuperKEKBの陽電子リングLERの運動量圧縮率は$\alpha_p = 3.2 \times 10^{-4}$なので、
  %
  \begin{equation}
    \gamma_t = \frac{1}{\sqrt{\alpha_p}}= \frac{1}{\sqrt{3.2 \times 10^{-4}}} = 55.9 \notag
  \end{equation}
  %
  陽電子の静止質量は$E_0 = \SI{0.5109989461}{\mega\electronvolt}$だから
  \begin{equation}
    E = \gamma_t E_0 = \SI{28.6}{\mega\electronvolt} \notag
  \end{equation}
  %
  となり、入射器からの入射エネルギー$\SI{4}{\giga\electronvolt}$の方がtransition energyより十分高い。
\end{tcolorbox}

\clearpage

\appendix
\renewcommand{\theequation}{\Alph{section}.\arabic{equation} }
\setcounter{equation}{0}

\section{相対論のおさらい}
静止エネルギー$E_0$, 静止質量$m_0$, 光速$c$, 運動エネルギー$E_k$, 全エネルギー$E$として、
%
\begin{equation}
  \begin{split}
    E_0 = m_0 c^2 ,\quad E = E_0 + E_k = mc^2\\
    \beta =\frac{v}{c}, \quad \gamma = \frac{E}{E_0}=\frac{m}{m_0}=\frac{1}{\sqrt{1-\beta^2}}
    \label{rest_energy}
  \end{split}
\end{equation}
%
また、運動量$p$とエネルギー$E$は
\begin{align}
  p &= mv = \gamma m_0 v = \gamma m_0 \beta c \label{momentum} \\
  E &= \gamma E_0 = \gamma m_0 c^2 \label{energy}
\end{align}
%
これより、
%
\begin{equation}
  \frac{p}{E} = \frac{\gamma m_0 \beta c}{\gamma m_0 c^2} = \frac{\beta}{c}
\end{equation}
%
(\ref{rest_energy})より、
%
\begin{equation}
  \beta = \sqrt{1-\frac{1}{\left(1+ E_k/E_0\right)^2}}
\end{equation}
%
電子の静止質量$E_0 = \SI{0.511}{\mega\electronvolt}$, 陽子の静止質量$E_0 = \SI{938}{\mega\electronvolt}$であるので、運動エネルギーとそれぞれの粒子の速度の関係をプロットすると、

\begin{figure}[hhbt]
  \begin{center}
    \includegraphics[width=12cm,clip]{figs/velocity.pdf}
    \caption{電子と陽子の運動エネルギーと速度の関係.}
    \label{velocity}
  \end{center}
\end{figure}

このグラフより、電子の速度は十分低いエネルギーでほぼ光速になっていることが分かる。
%
\paragraph{(\ref{delta_v}) の導出} \leavevmode\\

(\ref{momentum}) より、$p$を$v$で微分すると
%
\begin{equation}
  \frac{dp}{dv} = m_0\frac{d}{dv}(\gamma v)
  = m_0 \left(\gamma + v \frac{d\gamma}{dv}\right) \notag
\end{equation}
%
(\ref{rest_energy}) より、
%
\begin{align}
  \frac{d\gamma}{dv} & = \frac{1}{c}\frac{d\gamma}{d\beta}= \frac{1}{c}\frac{d}{d\beta}\left(\frac{1}{\sqrt{1-\beta^2}}\right) \notag \\
  & = \frac{1}{c} \beta \underset{\gamma^{-2}}{\uwave{(1-\beta^2)}}^{-\frac{3}{2}} = \frac{\beta \gamma^3}{c} \notag
\end{align}
%
これより、
\begin{align}
  \frac{dp}{dv} & = m_0 \left(\gamma + v \frac{\beta \gamma^3}{c}\right)
  = m_0 \gamma \underset{\gamma^2}{\uwave{(1 + \beta^2 \gamma^2)}}
  = \frac{\gamma^2 p}{v} \notag
\end{align}
%
したがって、
%
\begin{equation}
  \quad \frac{dv}{v} = \frac{1}{\gamma^2}\frac{dp}{p}
  \label{dv_dp}
\end{equation}
%
\paragraph{(\ref{delta_p})の導出}
%
\begin{align}
  p = mv = \gamma m_0 \beta c\,  , \quad E = \gamma E_0 = \gamma m_0 c^2 \notag
\end{align}
%
\begin{equation}
  \gamma \beta = \gamma \sqrt{1-\frac{1}{\gamma^2}} \notag = \sqrt{\gamma^2 -1} \notag
\end{equation}
%
\begin{align}
  \frac{dp}{dE} & = \frac{1}{c}\frac{d(\gamma \beta)}{d\gamma} = \frac{1}{c}\frac{d}{d\gamma}\sqrt{\gamma^2 -1} \notag \\
  & = \frac{\gamma}{c} \underset{\gamma^2\beta^2}{(\uwave{\gamma^2 -1}})^{-\frac{1}{2}} = \frac{1}{c\beta}
  = \frac{p}{\beta^2 E}\notag
\end{align}
%
したがって、
%
\begin{equation}
  \quad \frac{dp}{p} = \frac{1}{\beta^2}\frac{dE}{E}
  \label{dp_de}
\end{equation}

%
\begin{thebibliography}{9}
  \bibitem{McMillan}
  E.M. McMillan, Phys. Rev., 68, 143 (1945); V.I. Veksler, Compt. Rend. Acad. Sci. U.S.S.R., 43, 329 (1944); 44, 365 (1944).

\end{thebibliography}
%
\end{document}